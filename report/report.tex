\documentclass[a4paper,12pt]{report}

\usepackage{alltt, fancyvrb, url}
\usepackage{graphicx}
\usepackage[utf8]{inputenc}
\usepackage{float}
\usepackage{xcolor}
\usepackage{hyperref}

% Questo commentalo se vuoi scrivere in inglese.
\usepackage[italian]{babel}

\usepackage[italian]{cleveref}

\title{Relazione Temple-Tower per\\``Programmazione ad Oggetti''}

\author{Davide Vignali, Marko Cobo, Mattia Mularoni, Nicolas Montanari}
\date{\today}


\begin{document}

\maketitle

\tableofcontents

\chapter{Analisi}

\section{Descrizione e requisiti}

Il progetto Temple Tower si ispira ai classici dungeon crawler, offrendo un’esperienza di gioco a livelli in cui il giocatore esplora piani di un dungeon circolare per raccogliere tesori, sconfiggere nemici e salire verso il livello successivo. Il gioco culmina in un epico scontro con un boss finale.
%
La meccanica dei livelli circolari prende ispirazione dal gioco \href{https://ring-of-pain.fandom.com/wiki/Ring_of_Pain_Wiki}{Ring of Pain}.


\subsection*{Requisiti funzionali}
\begin{itemize}
	\item L'area di gioco è circolare ed è composta da diverse caselle contententi gli elementi di gioco che possono essere positivi o negativi per il giocatore.
	\item Gli elementi di gioco sono rappresentati da tesori (possono contenere punti esperienza, armi), trappole (tolgono punti vita), scale (permettono il passaggio a un livello superiore).
	\item Il giocatore si può muovere all'interno del livello, può combattere contro i nemici,  interagire con gli elementi di gioco.
	\item Dopo ogni scontro con un nemico o dopo aver attraversato una trappola il giocatore perde i punti vita, al termine dei quali la partita termina e si ritorna alla schermata inziale.
    \item Generazione casuale della torre per permettere partite sempre diverse.
\end{itemize}

\subsection*{Requisiti non funzionali}
\begin{itemize}
	\item Una volta arrivato all'ultimo piano il giocatore incontrerà il boss finale, il quale avrà comportamenti  unici e una difficoltà maggiore rispetto ai nemici normali .
	\item Difficoltà variabile in base al progresso di gioco.
	\item Musica di sottofondo e feedback sonori associati alle varie azioni.
	\item Esistono più tipologie di armi la quale efficacia varia rispetto al tipo di nemico.
\end{itemize}

\section{Modello del Dominio}
\textbf{Temple Tower} sarà rappresentato da una torre composta da un certo numero di piani (\textit{Floor}), 
%
ciascuno dei quali conterrà stanze collegate tra loro. Ogni stanza potrà contenere una trappola, un nemico, un tesoro e ospitare il giocatore.

\begin{itemize}
    \item \textbf{Trappole}: infliggono danni al giocatore.
    \item \textbf{Nemici}: attaccano il giocatore in combattimenti a turni.
    \item \textbf{Tesori}: forniscono punti vita, nuove armi, oppure potrebbe essere una trappola.
\end{itemize}
%
Il giocatore potrà utilizzare armi per affrontare i nemici. Poiché gli attacchi avvengono a turni, 
%
sarà fondamentale bilanciare le statistiche per evitare vittorie "matematiche" dovute a differenze di efficacia delle armi. 
%
Il giocatore e i nemici avranno a disposizione diverse tipologie di mosse di attacco e armi, che possono variare in potenza ed effetto.
%
Dopo aver esplorato le stanze di un piano, il giocatore potrà utilizzare le scale per salire al piano successivo della torre. 
%
Durante la partita, sia il giocatore sia i nemici avranno delle barre di stato: \textbf{vita}.
%
\begin{itemize}
    \item \textbf{Barra della vita}: si riduce subendo danni da nemici o trappole. Quando arriva a zero, la partita termina e si ricomincia dall'inizio.
\end{itemize}
%
All’ultimo piano, il giocatore affronterà il \textbf{boss finale}, un nemico più forte, dotato di mosse di attacco avanzate e particolari rispetto ai nemici ordinari.
%
Per arricchire l’esperienza, sarà presente un sottofondo musicale durante tutto il gioco.
%
\begin{figure}[H]
	\includegraphics[scale=0.15]{img/analisys.png}
\end{figure}

\chapter{Design}

\section{Architettura}

L'architettura del gioco \textbf{Temple Tower} segue il pattern architetturale \textbf{Model-View-Controller (MVC)} per garantire una chiara separazione delle responsabilità tra la logica di business, la presentazione e la gestione degli eventi.

\subsection*{Composizione del Pattern MVC}
\begin{itemize}
    \item \textbf{Model}: 
    \begin{itemize}
        \item Rappresenta la logica principale del gioco e include classi come \texttt{Tower}, \texttt{Floor}, \texttt{RoomBehavior}, \texttt{Player}, e i vari tipi di stanze (\texttt{EnemyRoom}, \texttt{TreasureRoom}, \texttt{StairsRoom}) che implementano il pattern Strategy.
        \item Questo approccio consente di definire comportamenti specifici per ogni tipologia di stanza in modo modulare, rendendo semplice l'aggiunta di nuovi tipi di stanze senza modificare il codice esistente.
        \item La logica di gioco, come il movimento del giocatore o gli effetti delle interazioni con nemici, trappole o tesori, è interamente contenuta nel model.
    \end{itemize}

    \item \textbf{View}: 
    \begin{itemize}
        \item Tutta la view è gestita da uno (\texttt{SceneManager}), il quale si occupa, mediante l'implementazione di un pattern Factory, di gestire il
        %
		cambiamento della vista in base alla situazione attuale.
        \item Grazie al manager è possibile decentralizzare le responsabilità della vista a n classi, le quali si occuperanno della gestione degli eventi
        %
		della singola interfaccia.
    \end{itemize}

    \item \textbf{Controller}: 
    \begin{itemize}
        \item La classe \texttt{GameController} funge da intermediario tra il modello e la vista, orchestrando il flusso degli eventi nel gioco.
        \item Gestisce le azioni dell'utente, permette al giocatore di muoversi tra le stanze (\texttt{changeRoom}) o salire al piano successivo (\texttt{gotoNextFloor}), si occupa inoltre di iniziare e terminare il gioco.
    \end{itemize}
\end{itemize} 

\subsection*{Scalabilità e Manutenibilità}
Grazie all'uso combinato dei pattern MVC e Strategy:
\begin{itemize}
    \item \textbf{Aggiunta di nuove stanze}: È possibile introdurre nuove tipologie di stanze semplicemente aggiungendo nuove implementazioni dell'interfaccia \texttt{RoomBehavior}, senza modificare altre parti del codice.
    \item \textbf{Separazione delle responsabilità}: La gestione della logica di gioco, della presentazione grafica e delle interazioni dell'utente è ben separata, favorendo la manutenibilità e la possibilità di cambiare singole componenti senza influenzare le altre.
\end{itemize}

Questa architettura rende il sistema flessibile, modulare e facilmente estensibile, adattandosi alle necessità di futuri miglioramenti o aggiunte.
%
\begin{figure}[H]
	\includegraphics[scale=0.12]{img/design.png}
\end{figure}

\section{Design dettagliato}
\subsection{Vignali}

\subsubsection{Modular Game Data Loading and Tower Configuration}

\paragraph{Problema} Il sistema necessita di caricare e gestire dati di gioco da file JSON esterni in modo flessibile e modulare, permettendo l'uso sia di torri predefinite che di mod create dagli utenti. I dati devono essere validati e supportare riferimenti relativi tra file.

\paragraph{Soluzione} Ho valutato due approcci principali:
\begin{enumerate}
    \item Un sistema di caricamento dati decentralizzato dove ogni componente carica i propri dati
    \item Un gestore centralizzato che coordina tutto il caricamento
\end{enumerate}
Ho scelto la seconda opzione implementando GameDataManagerImpl come Singleton per garantire un punto di accesso unico ai dati di gioco. Ho considerato l'uso di Dependency Injection, ma data la natura globale dei dati di gioco e la necessità di garantire uno stato coerente, Singleton è risultato più appropriato.

Il sistema utilizza Record immutabili (FloorData, Enemy, Weapon) e custom deserializer GSON per garantire type-safety e validazione durante il caricamento. Questa soluzione permette una chiara separazione tra dati e logica, facilitando l'estensione del sistema con nuove mod.

Schema UML:
\begin{figure}[H]
	\centering
	\includegraphics[scale=0.08]{img/GameDataManagerUML.png}
	\caption{GameDataManagerImpl as Singleton and its relations.}
\end{figure}

\subsubsection{Sistema di Modding}

\paragraph{Problema} Il gioco deve permettere agli utenti di creare e importare torri personalizzate (mod) sia da cartelle che da file ZIP, gestendo validazione, conflitti di nomi e isolamento tra mod.

\paragraph{Soluzione} Ho considerato due possibili architetture:

\begin{enumerate}
    \item Un sistema event-based con chiamate asincrone per l'importazione
    \item Un'architettura MVC con Observer pattern per la sincronizzazione UI
\end{enumerate}
Ho scelto la seconda opzione perché offre una separazione più chiara delle responsabilità e una gestione più prevedibile dello stato. Il pattern Observer è stato preferito a un sistema di callback perché permette di aggiungere facilmente nuovi observer senza modificare il codice esistente.

Ho implementato anche il pattern Strategy per l'importazione, permettendo di aggiungere facilmente nuovi formati di mod oltre a ZIP e cartelle.

Schema UML:
\begin{figure}[H]
	\centering
	\includegraphics[scale=0.08]{img/ModdingMenuUML.png}
	\caption{Modding Menu UML: MVC and Observer pattern overview.}
\end{figure}

\subsubsection{Gestione dello Spawn System}

\paragraph{Problema} Il sistema deve generare contenuti procedurali bilanciati rispettando vincoli di livello e configurazioni di spawn.

\paragraph{Soluzione} Ho valutato due approcci:
\begin{enumerate}
	\item Generazione puramente casuale con post-validazione
	\item Sistema template-based con strategie di Generazione
\end{enumerate}

Ho scelto il secondo approccio implementando un Template Method in SpawnManagerImpl che standardizza il processo di generazione permettendo variazioni nel comportamento specifico. Questo pattern è stato preferito a una soluzione più flessibile ma potenzialmente caotica basata su eventi.
Il metodo template definisce tre fasi:
\begin{enumerate}
	\item Selezione del tipo di piano (template method)
	\item Generazione delle stanze (hook method)
	\item Popolamento delle stanze (hook method)
\end{enumerate}

\subsection{Other members}

In questa sezione si possono approfondire alcuni elementi del design con maggior dettaglio.
%
Mentre ci attendiamo principalmente (o solo) interfacce negli schemi UML delle sezioni precedenti,
in questa sezione è necessario scendere in maggior dettaglio presentando la struttura di alcune sottoparti rilevanti dell'applicazione.
%
È molto importante che, descrivendo la soluzione ad un problema, quando possibile si mostri che non si è re-inventata la ruota ma si è applicato un design pattern noto.
%
Che si sia utilizzato (o riconosciuto) o meno un pattern noto, è comunque bene definire qual è il problema che si è affrontato, qual è la soluzione messa in campo, e quali motivazioni l'hanno spinta.
%
È assolutamente inutile, ed è anzi controproducente, descrivere classe-per-classe (o peggio ancora metodo-per-metodo) com'è fatto il vostro software: è un livello di dettaglio proprio della documentazione dell'API (deducibile dalla Javadoc).

\textbf{È necessario che ciascun membro del gruppo abbia una propria sezione di design dettagliato,
di cui sarà il solo responsabile}.
%
Ciascun autore dovrà spiegare in modo corretto e giustamente approfondito (non troppo in dettaglio, non superficialmente) il proprio contributo.
%
È importante focalizzarsi sulle scelte che hanno un impatto positivo sul riuso, sull'estensibilità, e sulla chiarezza dell'applicazione.
%
Esattamente come nessun ingegnere meccanico presenta un solo foglio con l'intero progetto di una vettura di Formula 1, ma molteplici fogli di progetto che mostrano a livelli di dettaglio differenti le varie parti della vettura e le modalità di connessione fra le parti, così ci aspettiamo che voi, futuri ingegneri informatici, ci presentiate prima una visione globale del progetto, e via via siate in grado di dettagliare le singole parti, scartando i componenti che non interessano quella in esame.
%
Per continuare il parallelo con la vettura di Formula 1, se nei fogli di progetto che mostrano il
design delle sospensioni anteriori appaiono pezzi che appartengono al volante o al turbo, c'è una
chiara indicazione di qualche problema di design.

Si divida la sezione in sottosezioni, e per ogni aspetto di design che si vuole approfondire, si presenti:
\begin{enumerate}
    \item: una breve descrizione in linguaggio naturale del problema che si vuole risolvere, se necessario ci si può aiutare con schemi o immagini;
    \item: una descrizione della soluzione proposta, analizzando eventuali alternative che sono state prese in considerazione, e che descriva pro e contro della scelta fatta;
    \item: uno schema UML che aiuti a comprendere la soluzione sopra descritta;
    \item: se la soluzione è stata realizzata utilizzando uno o più pattern noti, si spieghi come questi sono reificati nel progetto
    (ad esempio: nel caso di Template Method, qual è il metodo template;
    nel caso di Strategy, quale interfaccia del progetto rappresenta la strategia, e quali sono le sue implementazioni;
    nel caso di Decorator, qual è la classe astratta che fa da Decorator e quali sono le sue implementazioni concrete; eccetera);
\end{enumerate}
%
La presenza di pattern di progettazione \emph{correttamente utilizzati} è valutata molto positivamente.
%
L'uso inappropriato è invece valutato negativamente: a tal proposito, si raccomanda di porre particolare attenzione all'abuso di Singleton, che, se usato in modo inappropriato, è di fatto un anti-pattern.

\subsection*{Elementi positivi}

\begin{itemize}
	\item Ogni membro del gruppo discute le proprie decisioni di progettazione, ed in particolare le azioni volte ad anticipare possibili cambiamenti futuri (ad esempio l'aggiunta di una nuova funzionalità, o il miglioramento di una esistente).
	\item Si mostrano le principali interazioni fra le varie componenti che collaborano alla soluzione di un determinato problema.
	\item Si identificano, utilizzano \textit{appropriatamente}, e descrivono diversi design pattern.
	\item Ogni membro del gruppo identifica i pattern utilizzati nella sua sottoparte.
	\item Si mostrano gli aspetti di design più rilevanti dell'applicazione, mettendo in luce la maniera in cui si è costruita la soluzione ai problemi descritti nell'analisi.
	\item Si tralasciano aspetti strettamente implementativi e quelli non rilevanti, non mostrandoli negli schemi UML (ad esempio, campi privati) e non descrivendoli.
	\item Ciascun elemento di design identificato presenta una piccola descrizione del problema calato
nell'applicazione, uno schema UML che ne mostra la concretizzazione nelle classi del progetto, ed
una breve descrizione della motivazione per cui tale soluzione è stata scelta, specialmente se è stato utilizzato un pattern noto. Ad esempio, se si
dichiara di aver usato Observer, è necessario specificare chi sia l'observable e chi l'observer; se
si usa Template Method, è necessario indicare quale sia il metodo template; se si usa Strategy, è
necessario identificare l'interfaccia che rappresenta la strategia; e via dicendo.
\end{itemize}

\subsection*{Elementi negativi}
\begin{itemize}
	\item Il design del modello risulta scorrelato dal problema descritto in analisi.
	\item Si tratta in modo prolisso, classe per classe, il software realizzato, o comunque si riduce la sezione ad un mero elenco di quanto fatto.
	\item Non si presentano schemi UML esemplificativi.
	\item Non si individuano design pattern, o si individuano in modo errato (si spaccia per design pattern qualcosa che non lo è).
	\item Si utilizzano design pattern in modo inopportuno. Un esempio classico è l'abuso di
Singleton per entità che possono essere univoche ma non devono necessariamente esserlo. Si rammenta
che Singleton ha senso nel secondo caso (ad esempio \texttt{System} e \texttt{Runtime} sono
singleton), mentre rischia di essere un problema nel secondo. Ad esempio, se si rendesse singleton
il motore di un videogioco, sarebbe impossibile riusarlo per costruire un server per partite online
(dove, presumibilmente, si gestiscono parallelamente più partite).
	\item Si producono schemi UML caotici e difficili da leggere, che comprendono inutili elementi di dettaglio.
	\item Si presentano schemi UML con classi (nel senso UML del termine) che ``galleggiano'' nello schema, non connesse, ossia senza relazioni con il resto degli elementi inseriti.
	\item Si tratta in modo inutilmente prolisso la divisione in package, elencando ad esempio le classi una per una.
\end{itemize}


\subsubsection{Cambio vista}

\begin{figure}[H]
\centering{}
\includegraphics[width=\textwidth]{img/scenemanager.png}
\caption{Rappresentazione UML del pattern Factory per la gestione delle viste (esempio con 2 viste)}
\label{img:factory}
\end{figure}

\paragraph{Problema} Il gioco ha diverse viste, una per ogni situazione (Menu iniziale, vista sulle stanze, stanze singole ecc...),
%
è necessario gestire la visualizzazione in un modo coerente.

\paragraph{Soluzione} Il sistema per la gestione delle viste utilizza il \textit{pattern Factory}, come da
\Cref{img:factory}: le classi che implementano le viste \texttt{NomevistaView} una volta create, vengono inserite nel mapping
%
interno di \texttt{SceneManager} il quale si occupa di: 
\begin{itemize}
	\item All'avvio del gioco: inizializzare le viste relative alle schermate inziali.
	\item Dopo il caricamento della torre: inizializzare le viste degli elementi di gioco.
	\item Quando si chiama \texttt{switchTo}: caricare il nuovo Pane nella scena e visualizzarlo.
\end{itemize}
Questa soluzione permette in ogni parte della view, richiamando \texttt{switchTo} di passare alla view successiva passando il nome di essa,
%
permettendo una certa flessibilità nella creazione e linking di eventuali nuove viste.

\subsubsection{Stanze del gioco}

\begin{figure}[H]
\centering{}
\includegraphics[width=\textwidth]{img/behavior.png}
\caption{Rappresentazione UML del pattern Strategy per le stanze del gioco}
\label{img:rooms}
\end{figure}

\paragraph{Problema} Temple Tower ha più stanze, ognuna con un contenuto differente.

\paragraph{Soluzione} Il sistema per la modellazione delle stanze utilizza il \textit{pattern Strategy}, come da
\Cref{img:rooms}: le implementazioni di \texttt{RoomBehavior} possono essere modificate, e a seconda di cosa si inserisce nel metodo
%
\texttt{interact()}, il giocatore subirà o lancierà (a seconda di \texttt{direction}) delle azione da/alla stanza.
%
Per esempio un metodo interact dentro Trap toglierà punti vita al Player, mentre la stanza Stairs cambierà il piano di quest'ultimo.
%
Questo pattern permette di ampliare i tipi possibili di stanze richiedendo poche modifiche all'interno del codice già esistente. 

\chapter{Sviluppo}
\section{Testing automatizzato}
Per il testing generale di Temple Tower è stato utilizzata la suite JUnit per il testing automatizzato.
\subsection*{Controller}
Sono state testate nel controller, tutte le classi che interagiscono tramite esso, nella view.

\begin{itemize}
 \item Giocatore: vita, cambio stanza e armi.
 \item Nemico: vita e attacchi.
 \item Cambio piano.
 \item Boss finale: sistema di decisione sull'arrivo alla stanza del boss finale.
\end{itemize}

\section{Note di sviluppo}

Questa sezione, come quella riguardante il design dettagliato va svolta \textbf{singolarmente da ogni membro del gruppo}.
%
Nella prima parte, ciascuno dovrà mostrare degli esempi di codice particolarmente ben realizzati,
che dimostrino proefficienza con funzionalità avanzate del linguaggio e capacità di spingersi oltre le librerie mostrate a lezione.

\begin{itemize}
	\item \textbf{Elencare} (fare un semplice elenco per punti, non un testo!) le feature \textit{avanzate} del linguaggio e dell'ecosistema Java che sono state
utilizzate. Le feature di interesse sono:
	\begin{itemize}
		\item Progettazione con generici, ad esempio costruzione di nuovi tipi generici, e uso di generici bounded.
		L'uso di classi generiche di libreria non è considerato avanzato.
		\item Uso di lambda expressions
		\item Uso di \texttt{Stream}, di \texttt{Optional} o di altri costrutti funzionali
		\item Uso di reflection
		\item Definizione ed uso di nuove annotazioni
		\item Uso del Java Platform Module System
		\item Uso di parti della libreria JDK non spiegate a lezione (networking, compressione, parsing XML, eccetera...)
		\item Uso di librerie di terze parti (incluso JavaFX): Google Guava, Apache Commons...
	\end{itemize}
	\item Si faccia molta attenzione a non scrivere banalità, elencando qui features di tipo ``core'', come le eccezioni, le enumerazioni, o le inner class: nessuna di queste è considerata avanzata.
	\item Per ogni feature avanzata, mostrata, includere:
	\begin{itemize}
		\item Nome della feature
		\item Permalink GitHub al punto nel codice in cui è stata utilizzata
	\end{itemize}
\end{itemize}

In questa sezione, \textit{dopo l'elenco},
vanno menzionati ed attributi con precisione eventuali pezzi di codice ``riadattati'' (o scopiazzati...) da Internet o da altri progetti,
pratica che tolleriamo ma che non raccomandiamo.
%
Si rammenta agli studenti che non è consentito partire da progetti esistenti e procedere per modifiche successive.
%
Si ricorda anche che i docenti hanno in mano strumenti antiplagio piuttosto raffinati e che ``capiscono'' il codice e la storia delle modifiche del progetto,
per cui tecniche banali come cambiare nomi (di classi, metodi, campi, parametri, o variabili locali),
aggiungere o togliere commenti,
oppure riordinare i membri di una classe vengono individuate senza problemi.
%
Le regole del progetto spiegano in dettaglio l'approccio dei docenti verso atti gravi come il plagiarismo.

I pattern di design \textbf{non} vanno messi qui.
%
L'uso di pattern di design (come suggerisce il nome) è un aspetto avanzato di design, non di implementazione,
e non va in questa sezione.

\subsection*{Elementi positivi}

\begin{itemize}
	\item Si elencano gli aspetti avanzati di linguaggio che sono stati impiegati
	\item Si elencano le librerie che sono state utilizzate
	\item Per ciascun elemento, si fornisce un permalink
	\item Ogni permalink fa riferimento ad uno snippet di codice scritto dall'autore della sezione (i docenti verificheranno usando \texttt{git blame})
	\item Se si è utilizzato un particolare algoritmo, se ne cita la fonte originale.
	Ad esempio, se si è usato Mersenne Twister per la generazione di numeri pseudo-random, si cita \cite{mersenne}.
	\item Si identificano parti di codice prese da altri progetti, dal web, o comunque scritte in forma originale da altre persone.
	In tal senso, si ricorda che agli ingegneri non è richiesto di re-inventare la ruota continuamente:
	se si cita debitamente la sorgente è tollerato fare uso di di snippet di codice open source per risolvere velocemente problemi non banali.
	Nel caso in cui si usino snippet di codice di qualità discutibile,
	oltre a menzionarne l'autore originale si invitano gli studenti ad adeguare tali parti di codice agli standard e allo stile del progetto.
	Contestualmente, si fa presente che è largamente meglio fare uso di una libreria che copiarsi pezzi di codice:
	qualora vi sia scelta (e tipicamente c'è), si preferisca la prima via.
\end{itemize}

\subsection*{Elementi negativi}
\begin{itemize}
	\item Si elencano feature core del linguaggio invece di quelle segnalate. Esempi di feature core da non menzionare sono:
    \begin{itemize}
        \item eccezioni;
        \item classi innestate;
        \item enumerazioni;
        \item interfacce.
    \end{itemize}
	\item Si elencano applicazioni di terze parti (peggio se per usarle occorre licenza, e lo studente ne è sprovvisto) che non c'entrano nulla con lo sviluppo, ad esempio:
    \begin{itemize}
        \item Editor di grafica vettoriale come Inkscape o Adobe Illustrator;
        \item Editor di grafica scalare come GIMP o Adobe Photoshop;
        \item Editor di audio come Audacity;
        \item Strumenti di design dell'interfaccia grafica come SceneBuilder: il codice è in ogni caso inteso come sviluppato da voi.
    \end{itemize}
	\item Si descrivono aspetti di scarsa rilevanza, o si scende in dettagli inutili.
	\item Sono presenti parti di codice sviluppate originalmente da altri che non vengono debitamente segnalate.
	In tal senso, si ricorda agli studenti che i docenti hanno accesso a tutti i progetti degli anni passati,
	a Stack Overflow,
	ai principali blog di sviluppatori ed esperti Java,
	ai blog dedicati allo sviluppo di soluzioni e applicazioni
	(inclusi blog dedicati ad Android e allo sviluppo di videogame),
	nonché ai vari GitHub, GitLab, e Bitbucket.
	Conseguentemente, è \emph{molto} conveniente \emph{citare} una fonte ed usarla invece di tentare di spacciare per proprio il lavoro di altri.
	\item Si elencano design pattern
\end{itemize}

\subsection{Vignali}
\subsubsection{Utilizzo della libreria gson}
Utilizzato principalmente in GameDataManager per l'importazione e il caricamento delle torri: \href{https://google.com}{permalink}/ 
\subsubsection{Utilizzo della libreria JavaFX}
Utilizato All'interno di ModdingMenuView insieme a codice CSS: \href{https://google.com}{permalink}/ 
\subsubsection{Utilizzo della libreria SL4J}
Utilizzato per la gestione dei log principalemente nei test ma anche in giro per il codice: \href{https://google.com}{permalink}/ 
\subsubsection{Utilizzo della libreria Apache Commons IO}
Utilizzato per la gestione dei file insime alla java util per l'importazione dei file: \href{https://google.com}{permalink}/ 
\subsubsection{Utilizzo della libreria Java util zip}
Utilizzato per la decompressione dei file: \href{https://google.com}{permalink}/ 
\subsubsection{Utilizzo di stream e lambda expressions}
Utilizzato in diverse parti qui un esempio: \href{https://google.com}{permalink}/ 
\subsubsection{Utilizzo di Optional}
Utilizzato in diverse parti qui un esempio: \href{https://google.com}{permalink}/ 
\subsection{Mularoni}
\subsubsection{Utilizzo di \texttt{Stream} e lambda expressions} - Esempio: \href{https://google.com}{permalink}/ 
\subsubsection{Utilizzo di \texttt{Threading e Task} in Javafx per prevenire race condition} - Esempio: \href{https://google.com}{permalink}/ 
\subsubsection{Utilizzo della libreria SLF4J} - Esempio: \href{https://google.com}{permalink}/ 
\subsection{Montanari}
\subsection{Cobo}

\subsection{Esempio}

\subsubsection{Utilizzo della libreria SLF4J}

Utilizzata in vari punti.
Un esempio è \url{https://github.com/AlchemistSimulator/Alchemist/blob/5c17f8b76920c78d955d478864ac1f11508ed9ad/alchemist-swingui/src/main/java/it/unibo/alchemist/boundary/swingui/effect/impl/EffectBuilder.java#L49}

\subsubsection{Utilizzo di \texttt{LoadingCache} dalla libreria Google Guava}

Permalink: \url{https://github.com/AlchemistSimulator/Alchemist/blob/d8a1799027d7d685569e15316a32e6394632ce71/alchemist-incarnation-protelis/src/main/java/it/unibo/alchemist/protelis/AlchemistExecutionContext.java#L141-L143}

\subsubsection{Utilizzo di \texttt{Stream} e lambda expressions}

Usate pervasivamente. Il seguente è un singolo esempio.
Permalink: \url{https://github.com/AlchemistSimulator/Alchemist/blob/d8a1799027d7d685569e15316a32e6394632ce71/alchemist-incarnation-protelis/src/main/java/it/unibo/alchemist/model/ProtelisIncarnation.java#L98-L120}

\subsubsection{Scrittura di metodo generico con parametri contravarianti}

Permalink: \url{https://github.com/AlchemistSimulator/Alchemist/blob/d8a1799027d7d685569e15316a32e6394632ce71/alchemist-incarnation-protelis/src/main/java/it/unibo/alchemist/protelis/AlchemistExecutionContext.java#L141-L143}

\subsubsection{Protezione da corse critiche usando \texttt{Semaphore}}

Permalink: \url{https://github.com/AlchemistSimulator/Alchemist/blob/d8a1799027d7d685569e15316a32e6394632ce71/alchemist-incarnation-protelis/src/main/java/it/unibo/alchemist/model/ProtelisIncarnation.java#L388-L440}


\chapter{Commenti finali}

In quest'ultimo capitolo si tirano le somme del lavoro svolto e si delineano eventuali sviluppi
futuri.

\textit{Nessuna delle informazioni incluse in questo capitolo verrà utilizzata per formulare la valutazione finale}, a meno che non sia assente o manchino delle sezioni obbligatorie.
%
Al fine di evitare pregiudizi involontari, l'intero capitolo verrà letto dai docenti solo dopo aver formulato la valutazione.

\section{Autovalutazione e lavori futuri}

\textbf{È richiesta una sezione per ciascun membro del gruppo, obbligatoriamente}.
%
Ciascuno dovrà autovalutare il proprio lavoro, elencando i punti di forza e di debolezza in quanto prodotto.
Si dovrà anche cercare di descrivere \emph{in modo quanto più obiettivo possibile} il proprio ruolo all'interno del gruppo.
Si ricorda, a tal proposito, che ciascuno studente è responsabile solo della propria sezione: non è un problema se ci sono opinioni contrastanti, a patto che rispecchino effettivamente l'opinione di chi le scrive.
Nel caso in cui si pensasse di portare avanti il progetto, ad esempio perché effettivamente impiegato, o perché sufficientemente ben riuscito da poter esser usato come dimostrazione di esser capaci progettisti, si descriva brevemente verso che direzione portarlo.

\subsubsection{Mularoni}
Considerando il tempo a disposizione e gli impegni derivanti da lavoro e altri esami, reputo questo progetto "accettabile" ma con un grande potenziale inespresso,
%
dovuto sia alla scarsa divisibilità del progetto a livello di idea (è difficile che tutti lavorino a compartimenti stagni) che di organizzazione generale.
\section{Difficoltà incontrate e commenti per i docenti}

Questa sezione, \textbf{opzionale}, può essere utilizzata per segnalare ai docenti eventuali problemi o difficoltà incontrate nel corso o nello svolgimento del progetto, può essere vista come una seconda possibilità di valutare il corso (dopo quella offerta dalle rilevazioni della didattica) avendo anche conoscenza delle modalità e delle difficoltà collegate all'esame, cosa impossibile da fare usando le valutazioni in aula per ovvie ragioni.
%
È possibile che alcuni dei commenti forniti vengano utilizzati per migliorare il corso in futuro: sebbene non andrà a vostro beneficio, potreste fare un favore ai vostri futuri colleghi.
%
Ovviamente \textit{il contenuto della sezione non impatterà il voto finale}.

\appendix
\chapter{Guida utente}

Capitolo in cui si spiega come utilizzare il software. Nel caso in cui il suo uso sia del tutto
banale, tale capitolo può essere omesso.
%
A tal riguardo, si fa presente agli studenti che i docenti non hanno mai utilizzato il software
prima, per cui aspetti che sembrano del tutto banali a chi ha sviluppato l'applicazione possono non
esserlo per chi la usa per la prima volta.
%
Se, ad esempio, per cominciare una partita con un videogioco è necessario premere la barra
spaziatrice, o il tasto ``P'', è necessario che gli studenti lo segnalino.

\subsection*{Elementi positivi}

\begin{itemize}
 \item Si istruisce in modo semplice l'utente sull'uso dell'applicazione, eventualmente facendo uso di schermate e descrizioni.
\end{itemize}

\subsection*{Elementi negativi}
\begin{itemize}
 \item Si descrivono in modo eccessivamente minuzioso tutte le caratteristiche, anche minori, del software in oggetto.
 \item Manca una descrizione che consenta ad un utente qualunque di utilizzare almeno le funzionalità primarie dell'applicativo.
\end{itemize}

\chapter{Esercitazioni di laboratorio}

In questo capitolo ciascuno studente elenca gli esercizi di laboratorio che ha svolto
(se ne ha svolti),
elencando i permalink dei post sul forum dove è avvenuta la consegna.
%
Questa sezione potrebbe essere processata da strumenti automatici,
per cui link a oggetti diversi dal permalink della consegna,
errori nell'email o nel nome del laboratorio possono portare ad ignorare alcune consegne,
si raccomanda la massima precisione.

\section*{Esempio}

\subsection{paolino.paperino@studio.unibo.it}

\begin{itemize}
 \item Laboratorio 04: \url{https://virtuale.unibo.it/mod/forum/discuss.php?d=12345#p123456}
 \item Laboratorio 06: \url{https://virtuale.unibo.it/mod/forum/discuss.php?d=22222#p222222}
 \item Laboratorio 09: \url{https://virtuale.unibo.it/mod/forum/discuss.php?d=99999#p999999}
\end{itemize}

\subsection{paperon.depaperoni@studio.unibo.it}

\begin{itemize}
 \item Laboratorio 04: \url{https://virtuale.unibo.it/mod/forum/discuss.php?d=12345#p123456}
 \item Laboratorio 05: \url{https://virtuale.unibo.it/mod/forum/discuss.php?d=22222#p222222}
 \item Laboratorio 06: \url{https://virtuale.unibo.it/mod/forum/discuss.php?d=99999#p999999}
 \item Laboratorio 07: \url{https://virtuale.unibo.it/mod/forum/discuss.php?d=22222#p222222}
 \item Laboratorio 08: \url{https://virtuale.unibo.it/mod/forum/discuss.php?d=99999#p999999}
 \item Laboratorio 09: \url{https://virtuale.unibo.it/mod/forum/discuss.php?d=22222#p222222}
 \item Laboratorio 10: \url{https://virtuale.unibo.it/mod/forum/discuss.php?d=99999#p999999}
 \item Laboratorio 11: \url{https://virtuale.unibo.it/mod/forum/discuss.php?d=22222#p222222}
\end{itemize}


\bibliographystyle{alpha}
\bibliography{13-template}

\end{document}
